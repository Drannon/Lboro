\documentclass[../main.tex]{subfiles}
\graphicspath{{\subfix{../UTIL/Figures}}}
\begin{document}

\begin{enumerate}
    \item Data availability
    \item Accessibility - of method
    \item Output Relevance - Not just question but general insight
    \item Question Relevance - Does this answer the question?
    \item Development - Does it develop output of another method?
    \item Facilitation - Does it enable another method?
    \item Supporting Literature - Is it backed up?
    \item Maturity - Is it established?
    \item Flexibility - Can it be made it fit?
    \item Redundance - Is it already covered?
\end{enumerate}

To decide on the systems thinking methods that will answer the questions set out in part 4, ten criteria have been derived that various methods will be tested against in part 3b.
Methods will be given numerical scores for each of these criteria, and those with the highest scores will be selected.

What is the availability of the data that the method requires?
Some methods require difficult-to-obtain information to give useful insight or to use the method entirely.
A method that can be performed with public information but would offer better insight with less accessible information are still viable, though will be penalised against those that are insightful with accessible data.

What is the accessibility of the method?
A large number of methods can be performed with a pencil and paper or with a tool in a free digital office suite.
Some may require tools only present in commercial office suites, and some require dedicated tools to perform.
Methods will be graded on if the tools they require are expensive, difficult to learn, the availability of alternatives (e.g. free and open source software), and whether the tool locks certain options behind a paid license.
The time taken to complete the method will also be considered; some methods will take a matter of minutes to produce useful insight, whereas some may take hours or days to be useful.
A shorter method will generally be considered more viable.

What is the relevance of the output to the research question(s)?
Some methods will produce insight that directly helps to answer one or more of the research questions.
These methods will be favoured over those that are not directly applicable.

Does the method support the output of another method?
Some methods will be able to reinforce or even challenge the findings of other methods.
Whilst not necessarily being the most useful method to answer a question, its outputs build the credibility of other findings and thus the recommendations that come out of them.

Does the method's outputs improve or entirely enable another method?
Some methods may require the findings from another method as inputs.
Others may find they can be used in isolation, but are more useful when using information gathered from another methods.
Enabling or improving another method will result in a higher score.

Is the method based on good supporting literature?
Systems thinking methods are normally introduced in academic journal articles, or by individuals that use them in their work (e.g. consultancy).
Methods will be judged on the quality of their foundational material.

How mature is the method?
Having a large body of evidence where the method has been used gives credence to the method and its usefulness.
Methods that are mature, have been used often, and have been revised will be preferred.

How flexible is the method to this area of research?
Many different disciplines use systems thinking techniques, from healthcare to aviation.
Moreover, techniques are often tailored towards different tasks, such as risk management or accident investigation.
Methods will be judged on how easily they can be adapted to fit this paper's specific requirements.
Methods that fit well without adaptation will be given the maximum score.

Is the method covered in its entirety by one or more methods, or does it do the work of multiple methods?
Some methods will provide similar insights to other methods, and some will cover the same ground as multiple other methods.
Redundant methods will be penalised, and methods that make multiple others redundant will be preferred to reduce complexity.
