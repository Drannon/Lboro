The recommendations put forward are generally low-risk ethically, as they are generally restructuring or educational.
There is a concern with infrastructure subsidy and de-regulation, however.
Locations and the impact of any infrastructure must be carefully considered, to not damage any habitats and to not disrupt residents unduly.
Any construction must go through some public consultation, to ensure that residents have agency on where this infrastructure is built.
Additionally, there is a concern for emissions when constructing the infrastructure.
Construction can be heavily polluting, so care must be taken to choose low-carbon or carbon neutral construction methods.

There is also risk in nationalisation of the water suppliers.
Water suppliers, not least Southern Water, are complex organisations with large amounts of asset and debt.
Managing this with taxpayer money may prove challenging, considering the many differing organisations that vie for their proportion of the budget.
It is important to not let large debt and operational overheads hamper the efficacy of any organisation, water supplier or not.
