\documentclass[../main.tex]{subfiles}
\graphicspath{{\subfix{../UTIL/Figures}}}
\begin{document}

To produce the five recommendations, this paper considers six research questions whose answers will inform the recommendations.
\begin{enumerate}
    \item What key stakeholders are in the Test and Itchen Management Catchment, and what are their worldviews?
        \begin{itemize}
            \item This is vital for understanding who will be affected by the recommendations, and in understanding any pushback that may arise as a result.
        \end{itemize}
    \item What subsystems are present in the system, and how do they interact with each other?
        \begin{itemize}
            \item This is to see what the major areas of interest are, and where the recommendations should target.
        \end{itemize}
    \item What elements of the system are most sensitive to intervention?
        \begin{itemize}
            \item Like above, this is to ascertain which elements or subsystems should be targeted for intervention.
        \end{itemize}
    \item How effective are current initiatives at improving water security, and what can be learnt from them?
        \begin{itemize}
            \item Much work has been done to manage Britain's water resilience, so understanding what work has already been done will provide solid ground for new recommendations.
        \end{itemize}
    \item What negative consequences are likely from interventions in sensitive areas?
        \begin{itemize}
            \item What are the key issues that may arise from interventions? This way they may be mitigated or the recommendation may be restructured to avoid consequences.
        \end{itemize}
    \item What are the greenhouse gas emittors in the system and in current initiatives, and how can they be mitigated?
        \begin{itemize}
            \item Considering the problem space through a net-zero lens requires an understanding of the current impacts on climate change and thus how they may be targeted to reduce emissions.
        \end{itemize}
\end{enumerate}

To answer these questions, the methods selected in Section 3b have been performed below.

Initially, a Pig Model\cite{GovernmentOfficeforScience2022} was created (see figure \ref{fig:T4PIG}) to satisfy question 1 - who are the key stakeholders and what are their worldviews?
The Pig Model was developed as a quick method of identifying key stakeholders in a system, and in understanding how they view the system of interest (SoI).
Designed to be performed individually or collaboratively, the Pig Model is used to understand "who to include in... collaborating community."
The stakeholders were identified in a brainstorming session following research (see section 1) and using background knowledge, before updating and revising.
In this instance, there are nine key stakeholders who will need to be considered in this report:

\begin{itemize}
    \item Government (National and regional)
    \item Southern Water
    \item Environment Agency
    \item Catchment Partnerships
    \item Wildlife
    \item General Public
    \item Agriculture
    \item Energy Suppliers
    \item Industry
\end{itemize}

\begin{figure}[H]
    \includesvg[width=.9\linewidth]{UTIL/Figures/Method-application-pig-model.svg}
    \caption[width=.5\linewidth]{Pig Model\cite{GovernmentOfficeforScience2022} instance for Water Resilience.}
    \label{fig:T4PIG}
\end{figure}

A large amount of the value of the Pig Model is in its creation; beyond background knowledge, stakeholders and views were identified during drafting.
Some of these, for instance Government interest, were previously not captured.
Originally this was discounted as most interaction with the management catchment is performed through governmental departments and Quasi Non-Governmental Organisations (QUANGOs) such as the Environment Agency.
However, in considering the purposes of the management catchment to the EA, it became clear that some important aspects are without its remit and are instead matters for national government.
An example of this is strategic resource.
Downstream of the River Test is Southampton Water, on the banks of which is Marchwood Military Port.
Marchwood is the home base of the Tide class tankers for the Royal Fleet Auxiliary\cite{NavyCommandFOISection2021}, and Marchwood is
also home to the British Army's 17 Port and Maritime Regiment\cite{BritishArmy2025}.

Additionally, some shareholders were known but viewpoints emerged from the analysis.
For example, the General Public were initially identified as stakeholders but seeing the management catchment (and the rivers within it) as "heritage" was previously not captured.
An example of this is the Itchen Navigation, a "straightened, controlled and diverted part of the River Itchen".
It has a history of over 150 years, and forms the basis of a walking trail promoted by Winchester City Council's Visit Winchester website\cite{VisitWinchester2024}.
It also provided insight into the overlap of views between different stakeholders.
Southern Water, Agriculture, Energy Suppliers, and Industry all see the management catchment as a facilitator of their businesses.
Some, like Southern Water, are direct in that they exist to supply water from this management catchment to customers.
Others, like the Energy Suppliers, are less direct in that they use the water from the management catchment to cool Marchwood Power Station\cite{MarchwoodPower2011}.

\medskip

Following this, a systemigram was drawn to explore the systems at play in the management catchment.
Dr. Raquel Radoman's PhD Thesis\cite{LampaçaVieiraRadoman2025} was used as a guide, along with a workshop at the 2019 SERC Research Review presented by Brian Sauser\cite{Sauser2019}.
The systemigram was constructed by defining root definitions in the form "What to do (P), How to do it (Q) and Why do it (R)?"\cite{Checkland2000}.
Sauser\cite{Sauser2019} talks about the systemigram as a storyboarding tool, with different scenes representing different interests and viewpoints.
Each scene was thus given a root definition and the systemigram is divided into these scenes.
The systemigram can be seen in figure \ref{fig:T4SGM}.

\begin{figure}[H]
    \includesvg[width=.9\linewidth]{UTIL/Figures/Untitled Diagram-Copy of Full-Picture-Full-Picture.svg}
    \caption[width=.5\linewidth]{Systemigram of the Test and Itchen Management Catchment. Blue is the mainstay, red regulatory and legislative, yellow economic, and green environmental interests.}
    \label{fig:T4SGM}
\end{figure}

The blue bubbles represent the mainstay of the Systemigram, or the main root definition driving the system (figure \ref{fig:T4SGMS}):\\
\textit{The Test and Itchen Management Catchment Area is defined as an administrative area, (P) containing the river Test and river Itchen, to regulate uses such as abstraction, monitoring, conservation, and recreation, (Q) by issuing abstraction licenses to water companies such as Southern Water, external companies, independent suppliers, and industry and agriculture, (R) which satisfies the demand of customers such as the public, industry, energy, commercial, and agriculture.}

\begin{figure}[H]
    \includesvg[width=.9\linewidth]{UTIL/Figures/Untitled Diagram-Copy of Full-Picture-Scene-Mainstay.svg}
    \caption[width=.5\linewidth]{Mainstay of the Systemigram}
    \label{fig:T4SGMS}
\end{figure}

The yellow bubbles represent the business and economic scene (figure \ref{fig:T4SGMS1}):\\
\textit{The Customers pay fees to Water Companies who pay outgoings such as maintenance, operating costs, business overheads, and dividends to shareholders who own the water companies, and they invest in Infrastructure to supply customers. The water comapies use abstraction licenses to perform abstraction from the river Test and river Itchen through infrastrcture to supply customers.}

\begin{figure}[H]
    \includesvg[width=.9\linewidth]{UTIL/Figures/Untitled Diagram-Copy of Full-Picture-Scene 1 - Business.svg}
    \caption[width=.5\linewidth]{Scene 1 - Business}
    \label{fig:T4SGMS1}
\end{figure}

The red bubbles represent the regulatory and legislative scene(figure \ref{fig:T4SGMS2}):\\
\textit{The customers form a perception of water companies driven by the performance and practices of the water companies, which influences votes for government, who passes legislation to control and give remit to regulators such as the Environment Agency, Natural England, and OFWAT, who monitor the river Test and river Itchen. The Environment Agency then issues Abstraction licenses.}

\begin{figure}[H]
    \includesvg[width=.9\linewidth]{UTIL/Figures/Untitled Diagram-Copy of Full-Picture-Scene 2 - Regulation.svg}
    \caption[width=.5\linewidth]{Scene 2 - Regulation}
    \label{fig:T4SGMS2}
\end{figure}

The green bubbles represent the environmental scene, which does not have its own figure as it is reliant on the business scene:\\
\textit{Customers have demand for water, which is increased by infrastructure that loses water through leaks, which reduces the level of the river Test and the river Itchen. Customers also drive global warming, which reduces the water leels of the rivers, and also increase demand.}\\

Generating the systemigram was a useful activity, as the separate scenes provide clear delimination between different interests and also show where these different interests form part of the core "story" that the diagram tells.
Beyond that, it distilling the root definitions down to a diagramatic form has proved helpful as a reference.
Moreover, the systemigram shows interconnections between different elements and scenes.
The systemigram itself is a system, with elements and subsystems within it, and this model provides insight on the full system.
From the diagram emerges a focus on abstraction licenses, as they are a key way that water companies and other abstractors are regulated.
It also captures the central role that the Environment Agency plays, as a result of the importance of these licenses.
There is however a long chain of elements from the customer's perception of water companies to the abstraction licenses, however, which may show a lack of customer agency.

Considering the economic scene, we also see that water companies are strongly incentivised to abstract and supply as much water as possible.
Water Companies have a number of outgoings, which include investment in infrastructure along with other overheads.
As a private company, Southern Water's purpose is to generate value for its shareholders, who are paid in dividends taken from profits.
As such, the systemigram shows that they are incentivised to invest as little as possible into their infrastructure to maximise their profits and thus dividend payouts.
They are also incentivised to charge as high a fee as possible for their services, something exacerbated by a lack of choice for the customer; water companies are geographically assigned, as opposed to the chosen supplier model used by other utilities.
These fees are however regulated by Ofwat, which is not captured here but is captured in other methods (see figure \ref{fig:T4CLD}).

Some expenditure on infrastructure is necessary, as infrastructure is required for water to be supplied.
A focus on minimum expenditure however results in minimally viable infrastructure, which leaks.
The systemigram shows that these leaks contribute ultimately to demand for water, which lowers the available water in the river.
This is monitored by regulators, which prompts lowering of abstraction limits and increses of minimum flow limits to water companies.
With less income from a reduction in water supply, companies may be then less incentivised to invest in their infrastructure, resulting in more leaks in a reinforcing loop.

The systemigram also captures the role that global warming plays in the system, acting as a driver of demand and also reducing the available water in the two rivers.
As global warming is itself driven by customer activity, it serves to continuously reduce supply and increase demand regardless of actions on abstraction.
As seen in figure \ref{fig:T4PIG}, this also has implications for the river's other uses, namely as a habitat for endangered species and as recreation and environment for local residents.\\

With an understanding of the major systems at play, and their interactions, this information was used to generated an Ishikawa Diagram\cite{Ishikawa1976} (also known as a fishbone diagram, see figure \ref{fig:T4FBD})

\begin{figure}[H]
    \includesvg[width=.9\linewidth]{UTIL/Figures/method-applications-export.svg}
    \caption[width=.5\linewidth]{Ishikawa Diagram showing causes of insufficient water supply.}
    \label{fig:T4FBD}
\end{figure}

The Ishikawa diagram was originally designed to see the root causes of an undesirable outcome in a manufacturing environment.
It has been adapted in this paper to consider the undesirable outcome of "Insufficient Water" in a more holistic way.
Four root causes are defined (Demand, Supply, Infrastructure, and Authority) and shown as branches, and then their causes are listed as branches from those.
This continues until a set of root causes are determined.
Further fidelity could be acheived, but for the purposes of insight a high-level investigation is sufficient.

As with the pig model, the creation of the diagram was educative; some of the lower level causes were revealed, such as the cost of living affecting the public's ability to repair leaking water fixtures, wasting water overall.
It also helped categorise different causes, which went on to inform the grouping of concerns in the Causal Loop Diagram (see figure \ref{fig:T4CLD})
A key takeaway from the diagram is the focus on climate change's effect in the "Demand" branch, affecting all of "Increased Usage", "Energy Demand", and "Agriculture".
Additionally, the diagram captures the conflicting purposes that the water suppliers face:
They are legally obligated to supply water and have a duty of care over their environment, but also have pressure from their shareholders to grow and generate value for them.

All of the above diagrams have been used to generate a causal loop diagram, following the 

\begin{figure}[H]
    \includesvg[width=.9\linewidth]{UTIL/Figures/cld.svg}
    \caption[width=.5\linewidth]{Causal Loop Diagram showing the interactions different elements of the system have on each other.}
    \label{fig:T4CLD}
\end{figure}

The CLD shows the interactions between different elements, the elements that are key in the system and those that are on the periphery, and can also be used to predict the behaviour that emerges from the system over time.
A key way of doing this is by looking at system archetypes, as detailed by Kim and Anderson\cite{Kim1998}.
When looking at figure \ref{fig:T4CLD}, two archetypes are apparent.
The first is a sort of "shifting the burden", concerning leaks, new infrastucture and repairing old infrastructure.
There is a time delay in creating new infrastructure, and this is also driven by the appetite that the water companies have for new infrastructure investment.
It is required, however, to reduce the leaking and thus reduce overall amount of water needed to supply the demand, as shown by loop B2.
Loop B1 instead offers a quicker and cheaper "fix" in repairing the aging infrastructure that is being used.
This seems attractive and on the surface has the same effect as new infrastructure.
Loop R1 shows the consequence of this however, as in the long term there is a reduction in a desire for new infrastructure, which then only serves to make repairs more attractive.
Old infrastructure can only be repaired, not improved easily, so this solution does not scale with the increased needs being driven by population growth and climate change.
As a result, there is an overall reduction in leaks, but the burden has shifted to an infrastructure network that cannot be fit for purpose without even more costly repairs and expansions, or entirely new infrastructure.

Another archetype that is obvious is the "drifting goals" archetype, where it is easier to reduce expectations and bend limits than actually achieve the goals that are set.
This can be seen with the abstraction limits:
They are dependent on the water that flows through the rivers, and they serve to moderate the abstraction (B4).
However, this leads to a gap between the amount of water demanded and the amount available.
This can be remedied with better infrastructure, efforts to reduce demand, or even a reduction in emissions and climate change in the long term, but it is easier to simply pressure these limits to be lowered so that more water can be abstracted.
This has been shown in Southern Water's request to abstract more than they are allowed to from the river Test\cite{SouthernWaterServicesLimited2025}.
The CLD also shows that the main method of regulating usage of the river is through abstraction licenses.
They serve to limit the amount that can be abstracted, and as such are the main controllable element from the Environment Agency's viewpoint.
This corroborates what was seen in the systemigram (figure \ref{fig:T4SGM}).
Fines are the instrument by which these licenses (and targets from Ofwat) are enforced, but then the diagram shows that there are many effects of fines that have negative consequences.
An example of this is loop R2, where poor performance sours public perception of the company, which drives legislative change and regulator involvement.
This results in fines, which increase the company's debt - something that is already an issue, considering the company's reduction in credit rating from BBB (negative outlook) to BBB- (negative outlook) by Fitch\cite{SouthernWaterLimited2025}.
Such fines then result in an increase in fees, which serve to further degrade public opinion.

Finally, a context diagram was drafted to understand what areas are within the System of Interest (SOI), and can thus be controlled, which can be influenced (wider SoI), and which are uncontrollable but are important to take note of (Environment).
Written from the point of view of the Environment Agency, this diagram can be seen in figure \ref{fig:T4SOI}.

\begin{figure}[H]
    \includesvg[width=.9\linewidth]{UTIL/Figures/Method-application-context-diag.svg}
    \caption[width=.5\linewidth]{A context diagram showing what is controllable (SoI), what can be influenced (wider SoI), and what is uncontrollable (Environment and Wider Environment)}
    \label{fig:T4SOI}
\end{figure}

Of note in this diagram is how mamy important elements of the system are not within the SoI.
This includes legislation, emissions, and public opinion.
This is due to a few factors:
The Enviornment Agency can only use powers granted to it by legislation (such as abstraction licenses) to manage the management catchment.
A large number of important elements in the wider SoI are goverened by such legislation, which must pass through the Houses of Parliament.
As these are voted on by members, there is no way to directly control whether it gets passed.
Instead, the Environment Agency (and by extension DEFRA) can influence this by controlling content and by managing members and their constituents.
Even abstraction limits are outside of control to an extent, as the drought order requests are directed to DEFRA and not to the Environment Agency.
Directly within control however are also elements such as publications and education, which can help to reduce wasted water and reduce the demand for water overall.

\medskip

With the insight gained from the above methods, the research questions can be answered:

\begin{enumerate}
    \item The key stakeholders in the management catchment are Southern Water, the Environment Agency, Customers of Southern Water, and wildlife
    \item Key subsystems are the demand for water and subsequent abstraction moderated by abstraction licenses, the infrastrcutrure used to abstract and deliver this water, the business model that Southern Water operates, and the effect that climate change has.
    \item Elements most sensitive are those best connected, such as fees, fines, abstraction rates, and demand.
    \item Current initiatives such as the Catchment Based Approach\cite{DEFRA2013}, which aims to grant agency to the public who otherwise are removed from water companies' decisions, show promise and target an issue highlighted in this paper. Other initiatives, such as the new reservoires being constructed, may prove very useful but the time they take may stop them from avoiding outcomes such as supply discontinuity in the short term.
    \item Impacting fines is likely to result in reduced investment, and higher fees, and with the taxpayer eventually funding these companies as Operators of Last Resort, may only serve to saddle a new publicly owned supplier with debt.
    \item Greenhouse gasses are produced from abstraction itself, but also from energy demand, growing population, and agriculture. Mitigations that can be enacted are education on water demand's influence or subsidy of clean energy and low-carbon methods of construction. 
\end{enumerate}
