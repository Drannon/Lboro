\documentclass[../main.tex]{subfiles}
\graphicspath{{\subfix{../UTIL/Figures}}}
\begin{document}

This paper considers six research questions.
\begin{enumerate}
    \item What key stakeholders are in the Test and Itchen Management Catchment, and what are their worldviews?
    \item What subsystems are present in the system, and how do they interact with each other?
    \item What elements of the system are most sensitive to intervention?
    \item How effective are current initiatives at improving water security, and what can be learnt from them?
    \item What negative consequences are likely from interventions in sensitive areas?
    \item What are the greenhouse gas emitters in the system and in current initiatives?
\end{enumerate}

Initially, a Pig Model\cite{GovernmentOfficeforScience2022} was created (see figure \ref{fig:T4PIG}).
The Pig Model was developed as a quick method of identifying key stakeholders in a system, and in understanding how they view the system of interest (SoI).

\begin{figure}[H]
    \includesvg[width=.9\linewidth]{UTIL/Figures/Method-application-pig-model.svg}
    \caption[width=.5\linewidth]{Pig Model\cite{GovernmentOfficeforScience2022} instance for Water Resilience.}
    \label{fig:T4PIG}
\end{figure}

A large amount of the value of the Pig Model is in its creation; beyond background knowledge, stakeholders and views were identified during drafting.
For example, the General Public were initially identified as stakeholders but seeing the management catchment (and the rivers within it) as "heritage" was previously not captured.
An example of this is the Itchen Navigation, a "straightened, controlled and diverted part of the River Itchen"\cite{VisitWinchester2024}
It also provided insight into the overlap of views between different stakeholders.
Some, like Southern Water, are directly reliant that they exist to supply water from this management catchment to customers.
Others, like the Energy Suppliers, are less directly reliant in that they use the water from the management catchment to cool Marchwood Power Station\cite{MarchwoodPower2011}.\\

Following this, a systemigram (figure \ref{fig:T4SGM}) was drawn to explore the systems at play in the management catchment.
Radoman's work\cite{LampaçaVieiraRadoman2025} was used as a guide, along with a workshop presented by Brian Sauser\cite{Sauser2019}.
The systemigram was constructed by defining root definitions "What to do (P), How to do it (Q) and Why do it (R)?"\cite{Checkland2000}.
Sauser\cite{Sauser2019} talks about the systemigram as a storyboarding tool, with different scenes representing different interests and viewpoints.

\begin{figure}[H]
    \includesvg[width=.9\linewidth]{UTIL/Figures/Untitled Diagram-Copy of Full-Picture-Full-Picture.svg}
    \caption[width=.5\linewidth]{Systemigram of the Test and Itchen Management Catchment. Blue is the mainstay, red regulatory and legislative, yellow economic, and green environmental interests.}
    \label{fig:T4SGM}
\end{figure}

\textit{The Test and Itchen Management Catchment Area is defined as an administrative area, (P) containing the river Test and river Itchen, to regulate uses such as abstraction, monitoring, conservation, and recreation, (Q) by issuing abstraction licenses to water companies such as Southern Water, external companies, independent suppliers, and industry and agriculture, (R) which satisfies the demand of customers such as the public, industry, energy, commercial, and agriculture.}

\begin{figure}[H]
    \includesvg[width=.9\linewidth]{UTIL/Figures/Untitled Diagram-Copy of Full-Picture-Scene-Mainstay.svg}
    \caption[width=.5\linewidth]{Mainstay of the Systemigram}
    \label{fig:T4SGMS}
\end{figure}

\textit{The Customers pay fees to Water Companies who pay outgoings such as maintenance, operating costs, business overheads, and dividends to shareholders who own the water companies, and they invest in Infrastructure to supply customers. The water companies use abstraction licenses to perform abstraction from the river Test and river Itchen through infrastructure to supply customers.}

\begin{figure}[H]
    \includesvg[width=.9\linewidth]{UTIL/Figures/Untitled Diagram-Copy of Full-Picture-Scene 1 - Business.svg}
    \caption[width=.5\linewidth]{Scene 1 - Business}
    \label{fig:T4SGMS1}
\end{figure}

\textit{The customers form a perception of water companies driven by the performance and practices of the water companies, which influences votes for government, who passes legislation to control and give remit to regulators such as the Environment Agency, Natural England, and OFWAT, who monitor the river Test and river Itchen. The Environment Agency then issues Abstraction licenses.}

\begin{figure}[H]
    \includesvg[width=.9\linewidth]{UTIL/Figures/Untitled Diagram-Copy of Full-Picture-Scene 2 - Regulation.svg}
    \caption[width=.5\linewidth]{Scene 2 - Regulation}
    \label{fig:T4SGMS2}
\end{figure}

The green bubbles represent the environmental scene, which does not have its own figure as it is reliant on the business scene:\\
\textit{Customers have demand for water, which is increased by infrastructure that loses water through leaks, which reduces the level of the river Test and the river Itchen. Customers also drive global warming, which reduces the water levels of the rivers, and also increase demand.}\\

The systemigram itself is a system, with elements and subsystems within it, and this model provides insight on the full system.
From the diagram emerges a focus on abstraction licenses, as they are a key way that water companies and other abstractors are regulated.
It also captures the central role that the Environment Agency plays, as a result of the importance of these licenses.
There is however a long chain of elements from the customer's perception of water companies to the abstraction licenses, however, which may show a lack of customer agency.

As a private company, Southern Water's purpose is to generate value for its shareholders, who are paid in dividends taken from profits.
As such, the systemigram shows that they are incentivised to invest as little as possible into their infrastructure to maximise their profits and thus dividend payouts.
The systemigram also shows that leaks contribute ultimately to demand for water, which lowers the available water in the river.

The systemigram also captures the role that global warming plays in the system, acting as a driver of demand and also reducing the available water in the two rivers.
As global warming is itself driven by customer activity, it serves to continuously reduce supply and increase demand regardless of actions on abstraction.

With an understanding of the major systems at play, and their interactions, this information was used to generated an Ishikawa Diagram\cite{Ishikawa1976} (figure \ref{fig:T4FBD})

\begin{figure}[H]
    \includesvg[width=.9\linewidth]{UTIL/Figures/method-applications-export.svg}
    \caption[width=.5\linewidth]{Ishikawa Diagram showing causes of insufficient water supply.}
    \label{fig:T4FBD}
\end{figure}

The diagram has been adapted to consider the undesirable outcome of "Insufficient Water" in a holistic way.
A key takeaway from the diagram is the focus on climate change's effect in the "Demand" branch, affecting all of "Increased Usage", "Energy Demand", and "Agriculture".
Additionally, the diagram captures the conflicting purposes that the water suppliers face:
They are legally obligated to supply water and have a duty of care over their environment, but also have pressure from their shareholders to grow and generate value for them.

\begin{figure}[H]
    \includesvg[width=.9\linewidth]{UTIL/Figures/cld.svg}
    \caption[width=.5\linewidth]{Causal Loop Diagram showing the interactions different elements of the system have on each other.}
    \label{fig:T4CLD}
\end{figure}

The CLD (figure \ref{fig:T4CLD}) shows the interactions between different elements, and can also be used to predict the behaviour that emerges from the system over time.
A key way of doing this is by looking at system archetypes, as detailed by Kim and Anderson\cite{Kim1998}.
The first is "shifting the burden":
Loops B1 and B2 are both solutions to leaks in infrastructure.
There is a delay in creating new infrastructure, and this is driven by the appetite that the water companies have for new infrastructure investment.
B1 instead offers a quicker and cheaper "fix" in repairing the aging infrastructure that is being used.
R1 shows the consequence of this: a reduction in desire for new infrastructure, which then makes repairs more attractive.
This solution does not scale with increased needs being driven by population growth and climate change.
As a result, there is a short term reduction in leaks, but the burden has shifted to infrastructure that cannot be fit for purpose without expansions or entirely new infrastructure.

Another archetype is the "drifting goals" archetype.
This can be seen with the abstraction limits:
They are dependent on the water that flows through the rivers, and they serve to moderate the abstraction (B3).
This leads to a gap between the amount of water demanded and the amount available.
This can be remedied, but it is easier to simply pressure these limits to be lowered so that more water can be abstracted (R4).
This has been shown in Southern Water's request to abstract more than they are allowed to from the river Test\cite{SouthernWaterServicesLimited2025}.
The CLD also shows that the main method of regulating usage of the river is through abstraction licenses.
This corroborates what was seen in the systemigram (figure \ref{fig:T4SGM}).
Fines are the instrument by which these licenses (and targets from Ofwat) are enforced, but then the diagram shows that there are many effects of fines that have negative consequences.
An example of this is loop R2, where poor performance sours public perception of the company, which drives legislative change.
This results in fines, which increase the company's debt - something that is already an issue, considering the company's reduction in credit rating from BBB (negative outlook) to BBB- (negative outlook) by Fitch\cite{SouthernWaterLimited2025}.
This results in an increase in fees, which serve to further degrade public opinion.

Finally, a context diagram was drafted to understand what areas are within the System of Interest (SOI), and can thus be controlled, which can be influenced (wider SoI), and which are uncontrollable but are important to take note of (Environment).
Written from the point of view of the Environment Agency, this diagram can be seen in figure \ref{fig:T4SOI}.

\begin{figure}[H]
    \includesvg[width=.9\linewidth]{UTIL/Figures/Method-application-context-diag.svg}
    \caption[width=.5\linewidth]{A context diagram showing what is controllable (SoI), what can be influenced (wider SoI), and what is uncontrollable (Environment and Wider Environment)}
    \label{fig:T4SOI}
\end{figure}

Of note in this diagram is how many important elements of the system are not within the SoI.
This includes legislation, emissions, and public opinion.
This is due to a few factors:
A large number of important elements in the wider SoI are governed by such legislation, which must pass through the Houses of Parliament.
As these are voted on by members, there is no way to directly control whether it gets passed.
Directly within control however are also elements such as publications and education, which can help to reduce wasted water and reduce the demand for water overall.\\

With the insight gained from the above methods, the research questions can be answered:

\begin{enumerate}
    \item The key stakeholders in the management catchment are Southern Water, the Environment Agency, Customers of Southern Water, and wildlife
    \item Key subsystems are the demand for water and subsequent abstraction moderated by abstraction licenses, the infrastructure used to abstract and deliver this water, the business model that Southern Water operates, and the effect that climate change has.
    \item Elements most sensitive are those best connected, such as fees, fines, abstraction rates, and demand.
    \item Current initiatives such as the Catchment Based Approach\cite{DEFRA2013}, which aims to grant agency to the public who otherwise are removed from water companies' decisions, show promise and target an issue highlighted in this paper.
    \item Impacting fines is likely to result in reduced investment, and higher fees.
    \item Greenhouse gasses are produced from abstraction itself, but also from energy demand, growing population, and agriculture. 
\end{enumerate}

This investigation has lead to the following five recommendations:
\begin{enumerate}
    \item HM Government should subsidise the construction of new infrastructure, and remove some of the bureaucratic barriers to its construction.
        It is clear that the current system incentivises quick and cheap repairs over investment in the future, so making new infrastructure cheaper and less time-consuming will make it more appealing to water companies and will future-proof the infrastructure network.
        This will take time to see the impact for water security in the order of years, but in the short term increased construction effort will employ people and drive the economy, helping the government hit growth targets.
    \item HM Government should grant the Environment Agency more instruments to hold water companies accountable, and to give customers agency.
        There is a current plan to integrate parts of the Environment Agency and Ofwat, however the whole water regulation apparatus could be integrated into DEFRA to give it influence on policy and legislation.
        This body would also be able to remove reliance on fines, and they could introduce limits (not targets, routinely broken by Southern Water\cite{SouthernWaterServicesLimited2025}) on supply continuity and on emissions.
        This recommendation could be completed within the length of one Parliament (4 years) and benefits could be seen soon after this.
    \item The Environment Agency should continue the Catchment-Based Approach and further integrate it into the regulation apparatus.
        Catchment Partnerships are a vector by which communities can be involved with River Basin Management Plans\cite{DEFRA2013}.
        The benefits of this are already being seen, so further integration of it into the regulatory and management apparatus will have immediate positive effect.
    \item HM Government should consider taking Southern Water under public ownership, or operate it as a QUANGO.
        A similar initiative has been undertaken with train operating companies being nationalised upon the end of their contracts\cite{HMGovernment2025}.
        Doing this for water companies would remove the drive for profits and dividends from the system, removing the conflicting purposes and again giving residents more agency in their water supply (through elections etc.).
    \item The Environment Agency should be proactive in the education of the general public on the impact that their demand for water, food, and energy has on water resilience.
        As shown in figure \ref{fig:T4CLD}, climate change are driven by population - through food demand, energy demand, and water demand.
        Promote low-emission food choices such as plant-based diets, water saving techniques, and energy saving techniques to combat the effect that climate change is having on the management catchment.
\end{enumerate}
