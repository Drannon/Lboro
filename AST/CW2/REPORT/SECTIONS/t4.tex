\documentclass[../main.tex]{subfiles}
\graphicspath{{\subfix{../UTIL/Figures}}}
\begin{document}

To produce the five recommendations, this paper considers six research questions whose answers will inform the recommendations.
\begin{enumerate}
    \item What key stakeholders are in the Test and Itchen Management Catchment, and what are their worldviews?
        \begin{itemize}
            \item This is vital for understanding who will be affected by the recommendations, and in understanding any pushback that may arise as a result.
        \end{itemize}
    \item What subsystems are present in the system, and how do they interact with each other?
        \begin{itemize}
            \item This is to see what the major areas of interest are, and where the recommendations should target.
        \end{itemize}
    \item What elements of the system are most sensitive to intervention?
        \begin{itemize}
            \item Like above, this is to ascertain which elements or subsystems should be targeted for intervention.
        \end{itemize}
    \item How effective are current initiatives at improving water security, and what can be learnt from them?
        \begin{itemize}
            \item Much work has been done to manage Britain's water resilience, so understanding what work has already been done will provide solid ground for new recommendations.
        \end{itemize}
    \item What negative consequences are likely from interventions in sensitive areas?
        \begin{itemize}
            \item What are the key issues that may arise from interventions? This way they may be mitigated or the recommendation may be restructured to avoid consequences.
        \end{itemize}
    \item What are the greenhouse gas emittors in the system and in current initiatives, and how can they be mitigated?
        \begin{itemize}
            \item Considering the problem space through a net-zero lens requires an understanding of the current impacts on climate change and thus how they may be targeted to reduce emissions.
        \end{itemize}
\end{enumerate}

To answer these questions, the methods selected in Section 3b have been performed below.

Initially, a Pig Model\cite{GovernmentOfficeforScience2022} was created (see figure \ref{fig:T4PIG}) to satisfy question 1 - who are the key stakeholders and what are their worldviews?
The Pig Model was developed as a quick method of identifying key stakeholders in a system, and in understanding how they view the system of interest (SoI).
Designed to be performed individually or collaboratively, the Pig Model is used to understand "who to include in... collaborating community."
The stakeholders were identified in a brainstorming session following research (see section 1) and using background knowledge, before updating and revising.
In this instance, there are nine key stakeholders who will need to be considered in this report:

\begin{itemize}
    \item Government (National and regional)
    \item Southern Water
    \item Environment Agency
    \item Catchment Partnerships
    \item Wildlife
    \item General Public
    \item Agriculture
    \item Energy Suppliers
    \item Industry
\end{itemize}

\begin{figure}[H]
    \includesvg[width=.9\linewidth]{UTIL/Figures/Method-application-pig-model.svg}
    \caption[width=.5\linewidth]{Pig Model\cite{GovernmentOfficeforScience2022} instance for Water Resilience.}
    \label{fig:T4PIG}
\end{figure}

A large amount of the value of the Pig Model is in its creation; beyond background knowledge, stakeholders and views were identified during drafting.
Some of these, for instance Government interest, were previously not captured.
Originally this was discounted as most interaction with the management catchment is performed through governmental departments and Quasi Non-Governmental Organisations (QUANGOs) such as the Environment Agency.
However, in considering the purposes of the management catchment to the EA, it became clear that some important aspects are without its remit and are instead matters for national government.
An example of this is strategic resource.
Downstream of the River Test is Southampton Water, on the banks of which is Marchwood Military Port.
Marchwood is the home base of the Tide class tankers for the Royal Fleet Auxiliary\cite{NavyCommandFOISection2021}, and Marchwood is
also home to the British Army's 17 Port and Maritime Regiment\cite{BritishArmy2025}.

Additionally, some shareholders were known but viewpoints emerged from the analysis.
For example, the General Public were initially identified as stakeholders but seeing the management catchment (and the rivers within it) as "heritage" was previously not captured.
An example of this is the Itchen Navigation, a "straightened, controlled and diverted part of the River Itchen".
It has a history of over 150 years, and forms the basis of a walking trail promoted by Winchester City Council's Visit Winchester website\cite{VisitWinchester2024}.
It also provided insight into the overlap of views between different stakeholders.
Southern Water, Agriculture, Energy Suppliers, and Industry all see the management catchment as a facilitator of their businesses.
Some, like Southern Water, are direct in that they exist to supply water from this management catchment to customers.
Others, like the Energy Suppliers, are less direct in that they use the water from the management catchment to cool Marchwood Power Station\cite{MarchwoodPower2011}.



\begin{figure}[H]
    \includesvg[width=.9\linewidth]{UTIL/Figures/Untitled Diagram-Copy of Full-Picture.svg}
\end{figure}

\begin{figure}[H]
    \includesvg[width=.9\linewidth]{UTIL/Figures/method-applications-export.svg}
\end{figure}

\begin{figure}[H]
    \includesvg[width=.9\linewidth]{UTIL/Figures/Method-application-context-diag.svg}
\end{figure}

