\documentclass[../main.tex]{subfiles}
\graphicspath{{\subfix{../UTIL/Figures}}}
\begin{document}

The United Kingdom's Department for Environment, Food \& Rural Affairs (DEFRA) has divided England into ten River Basin districts, generally relating to their geographical location.
One of these is the South East district, which itself has 12 Management Catchments defined within it\cite{EACatchmentExplorerA}. 
This way of dividing England's water supply is the result of the Catchment Based Approach\cite{DEFRA2013}, enabling on "locally focussed decision making and action," supporting existing river basin management planning.
This approach defines a catchment as "A geographic area defined naturally by surface water hydrology.[W]e have adopted the definition of Management Catchments that the Environment Agency uses..."

One of these management catchments is the Test and Itchen Management Catchment, defined by the Environment Agency for water abstraction licensing\cite{EnvironmentAgency2019} as the catchments of both the River Test and the River Itchen in Hampshire.
The management catchment is predominatnly rural, comprising around $1760km^{2}$ of Hampshire.
The Test and Itchen are chalk streams, drawing flow from the groundwater along the northern section of the management catchment.\cite{EACatchmentExplorerB}
Both of the rivers have been declared Sites of Special Scientific Interest (SSSI) for their biodiversity\cite{NEDesignatedSitesViewA}\cite{NEDesignatedSitesViewB}, and the Itchen has been declared a Special Area of Conservation (SAC) for the presence of rare fauna \cite{NEDesignatedSitesViewC}.
The rivers 
