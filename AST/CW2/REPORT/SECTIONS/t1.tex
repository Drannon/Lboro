\documentclass[../main.tex]{subfiles}
\graphicspath{{\subfix{../UTIL/Figures}}}
\begin{document}

This essay sets out to provide five recommendations to \emph{regional governance} on improving water security in the Test and Itchen Management Catchment.
It investigates the management catchment for key stakeholders that must be considered in making recommendations, the system and subsystems at play in the region and their interactions, what lessons can be learnt from the current initiatives to improve water security, any knock on effects from interventions, and views the recommendations through a net-zero lens to assess their environmental impact.

The United Kingdom's Department for Environment, Food \& Rural Affairs (DEFRA) has divided England into ten River Basin districts, generally relating to their geographical location.
One of these is the South East district, which itself has 12 Management Catchments defined within it\cite{EACatchmentExplorerA}. 
This way of dividing England's water supply is the result of the Catchment Based Approach\cite{DEFRA2013}, enabling on "locally focussed decision making and action," supporting existing river basin management planning.
This approach defines a catchment as "A geographic area defined naturally by surface water hydrology.[W]e have adopted the definition of Management Catchments that the Environment Agency uses..."

One of these management catchments is the Test and Itchen Management Catchment, defined by the Environment Agency for water abstraction licensing\cite{EnvironmentAgency2019} as the catchments of both the River Test and the River Itchen in Hampshire (see figure~\ref{fig:MCA}).
The management catchment is predominantly rural, comprising around $1760km^{2}$ of Hampshire.
The Test and Itchen are chalk streams, drawing flow from the groundwater along the northern section of the management catchment.\cite{EACatchmentExplorerB}
Both of the rivers have been declared Sites of Special Scientific Interest (SSSI) for their biodiversity\cite{NEDesignatedSitesViewA}\cite{NEDesignatedSitesViewB}, and the Itchen has been declared a Special Area of Conservation (SAC) for the presence of rare fauna \cite{NEDesignatedSitesViewC}.

\begin{figure}[!htb]
    \includegraphics[width=.9\linewidth]{UTIL/Figures/test-and-itchen-mca.png}
    \caption[width=.8\linewidth]{Map of the Test and Itchen Managemenet Catchment\cite{ArcGIS:TIMCA}.}%
    \label{fig:MCA}
\end{figure}

The rivers serve as a major source of public water for Hampshire, with water being distributed throughout the management catchment as well as to other parts of Hampshire and to the Isle of Wight\cite{EACatchmentExplorerB}.
The majority of public water in the area is supplied by Southern Water with small sections of the management catchment being served by South West Water, Wessex Water, Thames Water, and South East Water\cite{ArcGIS:TIWaterCo} (see figure~\ref{fig:WCMCA}).
Additionally, the management catchment encompasses four large settlements: The cities of Southampton and Winchester, and the Towns of Andoever and Eastleigh.

\begin{figure}[!htb]
    \includegraphics[width=.9\linewidth]{UTIL/Figures/test-and-itchen-mca-water-co-ann.png}
    \caption[width=.8\linewidth]{Map of the Test and Itchen Managemenet Catchment, with Water Company shown \cite{ArcGIS:TIMCA}.}%
    \label{fig:WCMCA}
\end{figure}

Water availability and reliability in the Management Catchment are reported in the Environment Agency's Abstraction Licensing Strategy (ALS)\cite{EnvironmentAgency2019}, which dictates where individuals and organisations are granted licenses to abstract water from the rivers or groundwater.
It shows that from 18 test points along the Test and Itchen that 50\% (Q50) of the time, water is available in the majority of the catchment along the Test, but that along most of the Itchen and in Andover only restricted water is avaible (see figure~\ref{fig:ALSQ50}).
The availability in Andover and along the Itchen worsens to no water available when the flow is at levels exceeded 70\% of the time (Q70)(see figure~\ref{fig:ALSQ70}).

\begin{figure}[!htb]
    \includegraphics[width=.9\linewidth]{UTIL/Figures/ALS-Q50.png}
    \caption[width=.8\linewidth]{Water resource availability colours at Q50 for Test and Itchen ALS. Green: Water Available; Yellow: Restricted Water Available; Red: No Water Available.\cite{EnvironmentAgency2019}}%
    \label{fig:ALSQ50}
\end{figure}

\begin{figure}[!htb]
    \includegraphics[width=.9\linewidth]{UTIL/Figures/ALS-Q50.png}
    \caption[width=.8\linewidth]{Water resource availability colours at Q50 for Test and Itchen ALS. Green: Water Available; Yellow: Restricted Water Available; Red: No Water Available.\cite{EnvironmentAgency2019}}%
    \label{fig:ALSQ70}
\end{figure}
